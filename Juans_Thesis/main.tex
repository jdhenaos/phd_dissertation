%%
% This is an Overleaf template for Ph.D. theses
% using the TUM Corporate Desing https://www.tum.de/cd
%
% For further details on how to use the template, take a look at our
% GitLab repository and browse through our test documents
% https://gitlab.lrz.de/latex4ei/tum-templates.
%
% The tumbook class is based on the KOMA-Script class scrbook.
% If you need further customization please consult the KOMA-Script guide
% https://ctan.org/pkg/koma-script.
% Additional class options are passed down to the base class.
%
% If you encounter any bugs or undesired behaviour, please raise an issue
% in our GitLab repository
% https://gitlab.lrz.de/latex4ei/tum-templates/issues
% and provide a description and minimal working example of your problem.
%%


\documentclass[
  a4paper,            % paper size (a4paper, a5paper)
  thesis=phd,         % define the type of the thesis (student, phd, none)
  english,            % define the document language (english, german)
  % BCOR=5mm,           % define a binding offset for the document
  % coverBCOR=1cm,      % define a different binding offset for the cover page
  % coverpage=false,    % disable the cover page (e.g. if a tumcover is used)
  % titlepage=false,    % disable the additional title page
  % oneside,            % use onesided or twosided layout (oneside, twoside)
  % headmarks=true,     % enable headmarks (true, false)
  % font=times          % define main text font (helvet, times, palatino, libertine)
]{tum/tumbook}

% For theses that are printed with a transparent cover it is recommended to
% use the coverpage and provide the proper coverBCOR, so the distance between
% the binding strip and the content is properly set to 1 * logoheight.
% In this case, the publisher and titleback information is most certainly
% empty and the titlepage may be turned off.
%
% For theses that are printed with a soft cover or by a publisher it is
% recommended to create a cover using the tumcover class and therefore turn
% off the coverpage here. In this case, you most certainly have publisher and
% titleback information and you should keep the titlepage option enabled.


% load additional packages
\usepackage{lipsum}


% thesis metadata
\title{Thesis title}
\subtitle{Subtitle of the thesis}
\author{Author Name}

\degree{Doktor-Ingenieurs (Dr.-Ing.)}
\dateSubmitted{29.04.2016}
\dateAccepted{11.07.2016}

\committeeChair{Prof.\@ Franz~X.~Gabelsberger}
\committeeFirst{Prof.\@~Dr.\@ Georg Simon Ohm}
\committeeSecond{Prof.\@ James Clerk Maxwell}
% \committeeThird{optional}

\dedication{To Franz X. Gabelsberger, inventor of the street named after him.}

% publisher information for titlepage and titleback
% \publishers{Publisher}
% \uppertitleback{The uppertitleback usually contains\\
%   information about the author(s).}
% \lowertitleback{The lowertitleback contains ISBN, ISSN,\\
%   and other publisher information.}


\begin{document}

\frontmatter
\maketitle
\chapter{Abstract}
The abstract of your thesis goes here.

\lipsum[1-2]

\tableofcontents

\mainmatter
\chapter{Introduction}
This is the introduction of the thesis.

\lipsum[1]

\section{A section}
\lipsum[2]

\section{Another section}
\lipsum[3]


\chapter{Methodology}
This is the methodology of the thesis.

\lipsum[1]

\section{A section}
\lipsum[2]

\section{Another section}
\lipsum[3]

\chapter{Hyperoxia-induced cell cycle arrest drives long-term impairment of lung development and DNA repair in neonates}
This is the first project of the thesis.

\lipsum[1]

\section{A section}
\lipsum[2]

\section{Another section}
\lipsum[3]

\chapter{Multi-omic integration characterizes major endotypes of bronchopulmonary dysplasia linked to adult chronic lung disease}
This is the second project of the thesis.

\lipsum[1]

\section{A section}
\lipsum[2]

\section{Another section}
\lipsum[3]

\chapter{Multi-Omics Regulatory Network Inference in the Presence of Missing Data}
This is the third project of the thesis.

\section{Summary}

\section{Experimental set-up}
    \subsection{Data preprocessing}
    \subsection{Parameter selection}
    \subsection{Missingness simulated scenarios}
    \subsection{Performance evaluation criteria}
    
\section{Performance evaluation over different missingness scenarios}
    \subsection{Single-omics random missingness}
    \subsection{Multi-omics random missingness}
    \subsection{Block-wise random missingness}

\section{General method performance evaluation}
    \subsection{Performance evaluation over down-sampling}
    \subsection{Computational runtime evaluation}
    
\section{Discussion}
    \subsection{Limitations}
    \subsection{Outlooks}

\chapter{scNodes: Graphical Interface for Single-Cell Transcriptomics Workflow Development}
This is the forth project of the thesis.

\lipsum[1]

\section{A section}
\lipsum[2]

\section{Another section}
\lipsum[3]

\chapter{Cytokines Derived from Nasal Epithelial Lining Fluid in Patients with Asthma}
The fifth project.

\lipsum[1]

\section{A section}
\lipsum[2]

\section{Another section}
\lipsum[3]

\chapter{Discussion}
This is the discussion of the thesis.

\appendix
\chapter{Appendix}
\lipsum[4]

% \backmatter
% \bibliography{}

\end{document}
