\begin{enumerate}
	\item Juan Henao, Alida Kindt, Tanja Seegmüller, Kai Foerster, Andreas Flemmer3, Juergen Behr4, Nikolaus Kneidinger, Marion Frankenberger, Fabian Theis, Benjamin Schubert, Markus List, Anne Hilgendorff. Multi-omic signatures relate to the severity of pulmonary outcome in neonates traced into adult disease.\\
	
	\textbf{Summary}: This project focused on the detection of endotypes behind Bronchopulmonary Dysplasia (BPD) by proteomics, metabolomics, and clinical data integration using a cohort of 55 neonates with and without BPD. The endotypes were detected using Multi-Omics Latent Factor Analysis (MOFA) REF with sensitivity selection. We caught seven latent factors. However, none showed a sign of endotyping discrimination given the combined distribution of latent scores between no BPD and BPD patients in each latent factor. Nevertheless, the biological interpretation of each latent factor allowed us to discover a persistent inflammatory disease component in BPD.

We expanded our analysis by looking for individual molecular features with the potential to be biomarkers of severity using ANOVA with a t-test as a post-hoc method comparing no BPD, mild BPD, and moderate/severe BPD. Acknowledging the clinical heterogeneity signal of BPD cases, we reclassified them into no or moderate/severe BPD using a random forest model trained using oxygen supplementation and mechanical ventilation days (clinical variables used to diagnose BPD). We applied a t-test to identify significant molecular features between no BPD and moderate/severe BPD. We complement our analysis by training different random forest models combining significant molecular features and sets of increasing BPD characterization:

	\begin{enumerate}
		\item \textbf{BPD descriptors:} Oxygen supplementation and mechanical ventilation days.
		\item \textbf{Main risk variables:} Gestational age and birth weight.
		\item \textbf{Deep clinical phenotyping:} \textit{Main risk variables} and a compendium of clinical measurements encompassing comorbidities, medical interventions, and previously defined MRI-based scores.
	\end{enumerate}

The metabolite PC(O-36:5) was detected in both significant analyses and, combined with deep clinical phenotyping, improves the BPD classification along with PC(O-44:5) and gestational age. The protein CCL22 was detected in both significant analyses and improved the BPD classification according to random forest when combined with the main risk variables. Besides, SCGF-alpha, SCGF-beta, and KIR3DL2 were significantly different by ANOVA analysis of no, mild, and moderate/severe BPD comparison.

We traced our significant proteins in an adult chronic lung disease cohort composed of Chronic Obstructive Pulmonary Disease (COPD), Idiopathic Pulmonary Fibrosis (IPF), and healthy donors by ANOVA analysis comparing the three conditions. CCL22 and KIR3DL2 were detected in COPD, while SCGF-beta was significant in COPD and IPF. Those results support the hypothesis regarding the susceptibility of neonates with a BPD diagnosis to develop chronic lung diseases in adulthood.\\

	\textbf{Contribution:} I performed the data pre-processing and all the analyses used in this project. Besides, I created all the data visualization and wrote the first draft of the paper, which was reviewed and edited by Anne Hilgendorff, Markus List, and Tanja Segmuller.
	
	\item Erika Gonzalez Rodriguez1, Juan Henao2, Motaharehsadat Heydarian1, Tina Pritzke1, Alida Kindt3, Anna M. Dmitrieva1, Heiko Adler4, 5, Melanie Markmann6, Valeria Viteri-Alvarez1, Prajakta Oak1, Markus Koschlig1, Xin Zhang1, Kai M. Foerster7, Andreas Flemmer7, Hamid Hossain6,8, Xavier Pastor2, Holger Kirsten9, Peter Ahnert9, Juergen Behr10, Tushar J. Desai11, Benjamin Schubert2, Anne Hilgendorff1,12. Hyperoxia-induced cell cycle arrest drives long-term impairment of lung development and DNA repair in neonates.
	
	\item 	Juan David Henao Sanchez3,14, Mustafa Abdo1,2, MD, MSc, Benjamin Schubert3,14, PhD, Markus List4, PhD, Henrik Watz2,14, MD, Frauke Pedersen1,2,14, PhD, Alina Bauer3,15, MSc, Dominik Thiele5,14, MSc, Adam M. Chaker6,7, MD, Constanze A. Jakwerth 7,15, PhD, Benjamin Waschki1,8,14, MD, Anne Kirsten2,14, MD, Markus Weckmann9,14, PhD, Oliver Fuchs9,10,14, MD, PhD, Gesine Hansen11,16, MD, Matthias V. Kopp9,14, MD, Erika v. Mutius12,13,15, MD, MSc, Inke R. König4,14, PhD, Klaus F.  Rabe1,14, MD, PhD, Thomas Bahmer1,14, MD, Carsten B. Schmidt-Weber7,15, PhD, Ulrich M. Zissler7,15, PhD, and the ALLIANCE Study Group*. Cytokines Derived from Nasal Epithelial Lining Fluid in Patients with Asthma.
	
	\item Henao, J. D., Lauber, M., Azevedo, M., Grekova, A., Theis, F., List, M., ... \& Schubert, B. (2023). Multi-omics regulatory network inference in the presence of missing data. Briefings in Bioinformatics, 24(5), bbad309. 
\end{enumerate}