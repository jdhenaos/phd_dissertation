This is the introduction of the thesis.

\section{Pathophysiology of chronical lung diseases}
    \subsection{Bronchopulmonary Dysplasia (BPD)}
   Bronchopulmonary Dysplasia

Definition
Context
Bronchopulmonary Dysplasia (BPD) is the most common complication of prematurity, characterized by the impaired development of the gas exchange area and the bronchial tree.
Body
Wrap
Diagnosis + severity
Context (x)
The diagnosis of BPD has been dynamically changed over time since it was first reported in 1967 by Northway, Rosen, and Porter. However, the main clinical variables behind its diagnosis remain the basis complemented with new measurements based on recent insights about its onset.
Body (x)
BPD was first described as a hyaline membrane disease in preterm infants provoked by mechanical ventilation without any ending expiratory pressure and high levels of oxygen supplementation [REF]. In 2001, a workshop organized by the National Institute of Child Health and Human Development (NICHD), the National Heart, Lung, and Blood Institute (NHLBI), and the Office of Rare Diseases (ORD) (NICHD/NHLBI/ORD workshop), improved the diagnosis as well as providing severity classification based on the criteria described in Table XX.
Table XX. Diagnostic criteria of Bronchopulmonary Dysplasia*
	GA < 32 weeks	GA >= 32 weeks
Time point for assessment	36 weeks PMA of at discharge to home	> 28 days and <56 days postnatal age or at discharge to home
BPD diagnosis	Treatment with more than 21% oxygen concentration supplementation for at least 28 days
Mild BPD	Breathing room air at 36 weeks PMA or at discharge	Breathing room air by 56 days postnatal age or at discharge
Moderate BPD	Need of less than 30% oxygen concentration supplementation at 36 weeks PMA or at discharge	Need of less than 30% oxygen concentration supplementation at 56 days postnatal age or at discharge
Severe BPD	Need of more or equal than 30% oxygen concentration supplementation and/or positive pressure (PPV or NCPAP) at 36 weeks PMA or at discharge	Need of more than 30% oxygen concentration supplementation and/or positive pressure (PPV or NCPAP) at 56 days postnatal age at discharge
* Table taken and modified from REF
GA= gestational age; PMA= postmenstrual age; PPV= positive-pressure ventilation; NCPAP= nasal continuous positive airway pressure.

Although new modifications have been proposed recently, REF. The definition provided by NICHD/NHLBI/ORD workshop in 2001 has been widely used even nowadays to diagnose BPD.
Wrap (x)
Most of those proposed modifications remained restrictive to oxygen supplementation and mechanical ventilation usage or minimal addition of complementary clinical variables such as image-based readings REF. However, prenatal and postnatal risk variables have been widely studied as influential factors in triggering prematurity and potentially Bronchopulmonary Dysplasia as well.
Antenatal and postnatal risk factors
Context (x)
An effective and efficient diagnosis of premature neonates with BPD has been a central matter of study even in recent years REF. However, detecting risk factors is essential to predict potential susceptibility to BPD before the formal clinical diagnosis. Those risk factors do not include only parameters after neonate delivery (postnatal) but clinical variables during the pregnancy process (prenatal) REF.
Body (x)
Prematurity is by itself the major risk factor of BPD, which could be divided into gestational age (GA) and the birth weight of the neonate; the risk of developing BPD increases inversely proportionate with the increase in both factors REF. Furthermore, other factors such as intrauterine growth restriction (IUGR), male sex, chorioamnionitis (infection of the placenta and the amniotic fluid), race or ethnicity, and smoking have been defined as risk factors for BPD REF. Likewise, other pre and postnatal variables, including comorbidities, have been evaluated as potential risk factors, including maternal age, antenatal (prenatal) and postnatal corticosteroid administration, surfactant administration, inhaled nitric oxide therapy, patent ductus arteriosus (PDA; a neonatal cardiovascular condition), pulmonary hypertension, intraventricular hemorrhage (IVH; a brain-vascular condition), periventricular leukomalacia (PVL; a type of neonatal brain injury), and retinopathy of prematurity (ROP; a neonatal optical-vascular condition) REF.
Wrap (x)
The heterogeneity of clinical variables associated with BPD shed light on the complex molecular mechanisms behind the development and onset of neonatal chronic lung disease. Therefore, a deep molecular understanding, including the set of main biological pathways and molecular entities (genes, proteins and/or metabolites), seems necessary to improve the prevention and diagnosis powered by biomarkers and specific cellular-oriented treatments.
Molecular description
Context (x)
The molecular description of any disease becomes a pivot for efficient diagnosis and well-oriented treatments. In the case of BPD, the heterogeneous clinical landscape encompasses a set of biological pathways that could be grouped into inflammation, oxygen toxicity, growth factor signaling, and extracellular matrix-related processes. 
Body (x)
Inflammation is understood as a body’s immunity reaction mainly characterized by tissue swelling REF. In BPD, the inflammation occurs at the lung tissue level by the release of pro-inflammatory proteins called cytokines and the presence of pro-inflammatory cells like neutrophils and monocytes REF. Besides, adaptive immune cells like CD4+ T-cells have been observed to activate T-cells by decreasing the expression of CD62L in infants with BPD REF. Among the risk factors associated previously with BPD, chorioamnionitis has been associated with antenatal immune activation by increasing levels of cytokines such as IL-6, IL-8, and TNF-a in fetal circulation REF. In addition, antenatal lung inflammation impacts a variety of molecular regulatory pathways, such as toll-like receptors 2 and 4 (TLR2 and TLR4), growth factors like TGF-b and CTGF, and mesenchymal structural proteins like bone morphogenetic protein-4 inducing vascular remodeling and alveolar simplification, phenotypes associated to mild BPD onset REF. Likewise, oxygen supplementation could cause inflammation and oxygen toxicity when elevated concentrations (>21% FiO2) are used (hyperoxia) REF.

Hyperoxia is a well-known source of oxidative stress in premature neonates by the action of free radicals known as reactive oxygen species (ROS), which causes lung damage directly over epithelial and endothelial cells and proteins, metabolites, and nucleic acids destruction REF. ROS initiates apoptosis (programmed cell death) at the metabolic level through membrane lipids peroxidation, provoking the activation of the protein sphingomyelinase REF. The action of this protein releases large amounts of ceramides, which are the inductors of apoptosis REF. Likewise, cell death by ROS action could be achieved by activating proteins like key caspases and triggering receptor surface death receptors like Fas or mitochondrial cell death pathway activated by Bax proteins REF. On the other hand, ROS can directly oxidate nucleic acids, damaging the double-strain structure and causing cell death by necrosis or apoptosis REF. In BPD, increased levels of NOX1, a protein that produces superoxide radicals (a type of ROS), have been identified as a relevant participant in hyperoxia-induced acute oxidative stress injury, specifically by damaging the alveolar-capillary barrier REF. Both inflammation and hyperoxia-induced injuries affect internal cell functionality. However, extracellular processes could affect normal lung development in key regions such as alveolar and vascular tissues, inducing BPD REF.

The extracellular matrix (ECM) is the complex network of proteins and other molecules which offer structural and functional support to tissues REF. Collagen, the most abundant protein within the interstitial ECM, is composed mainly of collagen types I and III in the developing lung REF. Animal models have shown an increase in collagen 1 by the action of the transforming growth factor TGF-beta provoking thickened collagen fibres, increasing lung rigidity REF. This observation was confirmed through microscopic studies of BPD-diagnosed patients after positive-pressure ventilation REF. On the other hand, elastin, a component of elastic fibres in ECM, has been observed to decrease during impaired lung development, mainly affecting the correct alveolarisation process. Interestingly, the expression of the gene Eln (elastin) is stimulated by the expression of TGF-beta, which upregulation provokes rising levels of collagen 1 REF. 
Wrap (x)
The different efforts to understand the molecular mechanisms behind BPD have diverged in complex and heterogenic biological pathways that potentially hampered normal lung development, causing the most relevant lung phenotypes of BPD, gas exchange area and vascularity affection. However, most of the experiments have been carried out on animal models, delaying the discovery of molecular markers for an effective diagnosis and well-established treatment design.
Open problems
Context
Our increasing knowledge about BPD has improved the early treatment strategies efficiently, increasing the surveillance chances even in the most preterm infants (< 28 weeks PMA) REF. Nevertheless, this fact has opened new challenges, including dealing with comorbidities later in adolescence and/or adulthood REF. Furthermore, the current clinical-based BPD definition is not optimal for early diagnosis and severity prediction. 
Body
Nowadays, computational-based strategies are efficient and precise for prediction, treatment, and basic understanding-oriented tasks REF. The power of the current computational-based biomedical research is due to the capacity to join diverse types of data, including clinical records, diagnostic images (magnetic resonance image (MRI) or nuclear magnetic resonance (NMR), for example), and molecular data (e.g. gene or protein expression, and metabolite levels). This is possible thanks to the power of current machine learning and statistical techniques REF. 
Wrap



    \subsection{Asthma}
    \subsection{Chronic Obstructive Pulmonary Disease (COPD)}
    \subsection{Idiopathic Pulmonary Fibrosis (IPF)}

\section{Computational biology and chronic lung diseases}
    \subsection{Multi-omics data integration}
    \subsection{Clinical prediction}
    \subsection{Systems biology}
    
\section{Aims}
